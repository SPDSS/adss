\documentclass{memoir}

\usepackage[usenames, dvipsnames, svgnames, table]{xcolor}
\usepackage{tikz}
\usepackage[backgroundcolor = SkyBlue!25, bordercolor = SkyBlue, linecolor = SkyBlue, textsize = small]{todonotes}
\usepackage[colorlinks = true, urlcolor = RoyalBlue, citecolor = OrangeRed, linkcolor = RoyalBlue, hyperfootnotes = true]{hyperref}
\usepackage[margin = 25mm]{geometry}
\usepackage[scaled = 0.85]{DejaVuSansMono}
\usepackage{enumitem}

\pretitle{\centering\Huge\bfseries}
\posttitle{\par\vskip1em\begin{center}\normalfont\Large\textsc{ASPIRE Decision Support System}\vskip2em\includegraphics[width = 0.25\linewidth]{images/adss.png}\end{center}}
\preauthor{\vfill\scshape\centering}
\postauthor{\begin{center}\includegraphics[width = 0.25\linewidth]{images/polito.png}\end{center}}
\predate{\vfill\normalfont\centering}
\postdate{\vskip2em\begin{center}\includegraphics[width= 0.15\linewidth]{images/aspire.png}\end{center}\newpage}

\renewcommand\colorchapnum{\bfseries\hspace*{-2mm}}
\renewcommand\colorchaptitle{\normalfont\scshape}
\chapterstyle{pedersen}

\setlength{\marginparwidth}{20mm}

\newcommand{\PlugIn}[1]{\texttt{#1}}
\newcommand{\Method}[1]{\texttt{#1}}
\newcommand{\Class}[1]{\texttt{#1}}
\newcommand{\File}[1]{\texttt{#1}}
\newcommand{\Object}[1]{\texttt{#1}}

\title{ADSS -- Developer Manual}
\author{Politecnico di Torino}

\begin{document}
	\maketitle
	\tableofcontents*

	\newpage
	\vspace*{0em}
	\vfill
	\begin{center}
		\textsc{\Large{Disclaimer}}
		\vskip1em
		\fbox{\parbox{0.75\linewidth}{This manual documents the ASPIRE Decision Support System as released in date \today{} and its content it is likely to change as new versions are released.}}
	\end{center}
	\vfill

	\chapter{Preparing the development environment}
	
	The ADSS requires some preparations before a developer can start modifying it. It is mainly written in Java (or analogous technologies), so that it should be OS-independent.
	
	To setup a fully working development environment you need to follow the following initial steps:
	
	\begin{enumerate}
		\item install the following software into your system:
			\begin{itemize}
				\item JDK 7.x or superior (\url{http://www.oracle.com/technetwork/java/javase/downloads/});
				\item SWI-Prolog 7.2.x or superior (\url{http://www.swi-prolog.org});
			\end{itemize}
		\item install the following software in the ASPIRE virtual machine (or in your system if you want to do a local build):
			\begin{itemize}
				\item CodeSurfer 2.x or superior (with a proper builder license);
				\item the ASPIRE Annotation Extractor;
				\item the ACTC;
				\item at least one ASPIRE software protection tool.
			\end{itemize}
	\end{enumerate}
	
	After that you have two choices. The simplest one is to download an already packaged Eclipse, so that:
	
	\begin{itemize}
		\item if you are using Windows:
		\begin{enumerate}
			\item download the all-in-one Eclipse binaries from \url{x} and decompress it somewhere;
			\item download the ADSS workspace from \url{y} and decompress it somewhere else;
			\item launch Eclipse and when asked for the workspace directory choose your ADSS workspace location.
		\end{enumerate}
		\item if you are using Linux, wait a bit, someday somebody will put it online.
	\end{itemize}
	
	Alternatively, if you are brave enough, you can setup a working Eclipse and download the source from SVN:
	
	\begin{enumerate}
		\item download an Eclipse 4.x Mars for RCP and RAP Developers from \url{http://www.eclipse.org/} and install it;
		\item launch Eclipse and install the following plug-ins:
			\begin{itemize}
				\item from \url{http://download.eclipse.org/releases/mars} install:
				\begin{itemize}
					\item Subversive SVN Team Provider;
					\item EMF - Eclipse Modeling Framework Xcore SDK;
					\item EMF - Eclipse Modeling Framework SDK.
				\end{itemize}
				\item from \url{http://www2.imm.dtu.dk/~ekki/projects/ePNK/indigo/update/} install:
				\begin{itemize}
					\item ePNK Basic Extensions;
					\item ePNK Core;
					\item ePNK HLPNGs.
				\end{itemize}
				\item from \url{http://download.eclipse.org/nattable/releases/1.4.0/repository/} install everything;
				\item download \url{http://www2.cs.uni-paderborn.de/cs/kindler/research/EPCTools/downloads/epctools-2.0.3.plugin.eclipse-3.1.zip} and unzip it in you Eclipse plug-ins directory.
			\end{itemize}
		\item from Eclipse imports all the plug-ins from the SVN repository at \url{https://aspire-fp7.eu/framework/development/ADSS/} --- if asked for an SVN connector you can install SVNKit 1.8.x;
		\item create an Enclipse launch configuration using all the plug-ins in the workspace;
		\item for Windows only, modify the launch configuration in this way:
		\begin{itemize}
			\item add \texttt{-Djava.library.path="C:\\Program Files\\swipl\\bin"} to the VM arguments, specifying where you have installed SWI-Prolog binaries;
			\item add the \texttt{SWI\_HOME\_DIR} environment variable pointing to your SWI-Prolog directory (e.g. \texttt{C:\\Program Files\\swipl});
			\item add the \texttt{PATH} environment variable pointing to your SWI-Prolog binaries directory (e.g. \texttt{C:\\Program Files\\swipl\\bin}).
		\end{itemize}
	\end{enumerate}
	
	Remember to install Java, SWI-Prolog and Eclipse for 64 bit or for 32 bit OSses, without mixing them.
	
	Under a Debian/Ubuntu you can easily install the JRE and SWI-Prolog via the following command lines:
	
	\begin{quote}
		\texttt{aptitude install openjdk-8-jre openjdk-8-jre-headless}\\
		\texttt{aptitude install swi-prolog swi-prolog-java swi-prolog-nox}
	\end{quote}
	
	\chapter{ADSS plug-ins}

	The ADSS is split in several plug-ins, that are:
	
	\begin{itemize}
		\item the \PlugIn{eu.aspire\_fp7.adss}, developed by POLITO, contains most of the logic of the ADSS;
		\item the \PlugIn{eu.aspire\_fp7.adss.akb}, developed by POLITO, contains the AKB;
		\item the \PlugIn{eu.aspire\_fp7.adss.akb.ui}, developed by POLITO, contains the AKB editor UI;
		\item the \PlugIn{eu.aspire\_fp7.adss.help}, developed by POLITO, contains the ADSS help (and this manual);
		\item the \PlugIn{eu.aspire\_fp7.adss.ui}, developed by POLITO, contains some common ADSS UI classes;
		\item the \PlugIn{eu.aspire\_fp7.adss.util}, developed by POLITO, contains various utility methods;
		\item the \PlugIn{it.polito.security.ontologies}, developed by POLITO, contains the ontology API;
		\item the \PlugIn{org.uel.aspire.wp4.assessment}, developed by UEL, contains the Petri Nets API.
	\end{itemize}

	In addition you can also find two other (optional) plug-ins developed by POLITO:
	
	\begin{itemize}
		\item the \PlugIn{it.polito.security.ontologies.samples} contains some ontology examples;
		\item the \PlugIn{it.polito.security.ontologies.tools} contains some ontology analysis tools.
	\end{itemize}
	
	\chapter{Accessing and expanding the AKB}
	
	Most of the AKB is stored into a central ontology that can be easily accessed via the \Method{ADSS.getModel()} method. It returns a \Class{Model} EMF object that contains all the information used by the ADSS. It can be freely read and modified using a set of getters and setters. You can also perform a low-level access to the ontology by using the \Method{Model.getOntology()} method.
	
	The AKB \Class{Model} class contains the following principal getter methos:
	
	\begin{description}[style = nextline, before=\normalfont, font = \normalfont\ttfamily]
		\item[getOntology()] retrieves the ontology itself --- use it only if you are 100\% sure of what you are doing;
		\item[getApplications()] retrieves the list of software applications --- this method may be removed in the future;
		\item[getAttackPaths()] retrieves the list of detected attack paths;
		\item[getAttackSteps()] retrieves the list of detected attack steps;
		\item[getProtections()] retrieves the list of supported protections;
		\item[getAttacker()] retrieves the attacker profile, containing data such as the expertise level and his budget threshold;
		\item[getAttackerTools()] retrieves the list of supported attacker tools;
		\item[getSolutions()] retrieves the inferred solutions.
	\end{description}

	You can access a stable AKB in several points, such as:
	
	\begin{itemize}
		\item in the \Method{PetriNetsConnector.initialize()} method that it is called before actually calling the SPA tool to perform the Petri nets assessments;
		\item in the \Method{PetriNetsConnector.compare()} method that it is called when comparing two solutions in order to find the best one.
	\end{itemize}
	
	The developer can also choose to look into the following files for more information about what is inside the AKB:
	
	\begin{itemize}
		\item the file \File{owl/akb.owl} in the \PlugIn{eu.aspire\_fp7.adss.akb} plug-in. It contains the initial OWL ontology loadead when a new project is created;
		\item the file \File{xcore/akb.xcore} in the \PlugIn{eu.aspire\_fp7.adss.akb} plug-in. It contains the XCore model used to automatically generate the Java classes of the AKB. In here all the available getters and setters are also documented.
	\end{itemize}
	
%	\section{Adding new attack steps}
%	\section{Adding new attack tools}
%	\section{Adding new protections}

	\chapter{Modifying the UI}

	\section{Adding new steps to the build process}
	
	First, note that adding a new build step requires \emph{only} modifying the UI, so that your additional code can be placed anywhere. Basically, there are two things to do: modify the build all and step-by-step actions. To add a new step:
	
	\begin{enumerate}
		\item open the \Class{eu.aspire\_fp7.adss.akb.ui.editors.OverviewPage} class --- all the modifications must be performed here;
		\item locate the \Method{createAutomaticBuildSection} method and add the new code to be executed to the listener of the \Object{buildAllHyperlink} class field --- this will take care of the `Build All' hyper-link;
		\item create a new \Class{org.eclipse.ui.forms.widgets.ImageHyperlink} object as a new class field --- this widget will model the button used to launch the new step only;
		\item locate the \Method{createStepByStepBuildSection} method, add the code to allocate the new hyper-link and set-up its listener accordingly --- this will take care of the `Step by Step Build` hyper-link.
	\end{enumerate}

	\section{Adding a new page to the AKB editor}
	
	This modification is very straightforward. To add a new page:
	
	\begin{itemize}
		\item implement a new concrete class extending \Class{org.eclipse.ui.forms.editor.FormPage} containing the new custom content;
		\item open the \Class{eu.aspire\_fp7.adss.akb.ui.editors.AKBEditor} class --- this class model the AKB editor as a whole;
		\item modify the \Method{addPages} method in order to add the new page to the AKB editor.
	\end{itemize}

%	
%	\chapter{Modifying the solvers}
%	
%	\todo[inline]{Yet to do. I'm very lazy\dots}
%	
%	\chapter{Modifying the UI}
%	
%	\todo[inline]{Yet to do. I'm super lazy\dots}
%	
%	\chapter{API}
%
%	\todo[inline]{Yet to do. I'm extraordinarily lazy\dots}
\end{document}
